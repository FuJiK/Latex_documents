%全体の2段組.tex
\documentclass[twocolumn,10pt]{jarticle}
% 数式
\usepackage{amsmath,amsfonts}
\usepackage{bm}
% 画像
\usepackage[dvipdfmx]{graphicx}
\setlength{\columnsep}{3zw}
\title{\TeX とは}
\author{Name}
\date{\today}
%%%%%%    TEXT START    %%%%%%
% Latex Grammer http://www.ceser.hyogo-u.ac.jp/shino/tex/index.html
%文末の\ は補ってあげてね。\ \ に補って
\begin{document}
\maketitle
\section{\TeX について}
TeXはスタンフォード大学教授(数学)D.E.Knuth(19388~)による文書整形システムです。TeXは大抵「テフ」と読まれいます。TeXはワープロのたぐいと言えますが、より正しくは、1つのプログラム言語に近いものです。利用者によるマクロ命令によって機能を拡張することができます。今までは研究者の間でUNIX環境での稼働が一般的でしたが、今日では、個人がMacintoshOSやWindows9Xをインストールしたパーソナルコンピュータ上でTeXを動かすことが可能です。ネットワークで配布されているパッケージもありますが、最近では、安価にCD-ROMの形態で書籍に付録されているものもあり、ある程度の文法の理解は必要ですが、文書作成の種類や目的によっては、とても重宝なツールと言えます。……以下続く……
表環境の利用2
\begin{table}[h]
\begin{center}
\caption{表の番号は自動です}
\label{tbl:a1}
\begin{tabular}{|l|l|l|}
\hline
番号 & 氏名 & 適用 \\
\cline{1-3}
1 & 佐藤一郎 & ○○企画 \\
\hline
2 & 河野花子 & スタジオジャパン \\
\hline
3 & 奈良剛人 & (株)大和造園 \\
\hline
\end{tabular}
\end{center}
\end{table}
\twocolumn
TeXはスタンフォード大学教授(数学)D.E.Knuth(19388~)による文書整形システムです。TeXは大抵「テフ」と読まれいます。TeXはワープロのたぐいと言えますが、より正しくは、1つのプログラム言語に近いものです。利用者によるマクロ命令によって機能を拡張することができます。今までは研究者の間でUNIX環境での稼働が一般的でしたが、今日では、個人がMacintoshOSやWindows9Xをインストールしたパーソナルコンピュータ上でTeXを動かすことが可能です。ネットワークで配布されているパッケージもありますが、最近では、安価にCD-ROMの形態で書籍に付録されているものもあり、ある程度の文法の理解は必要ですが、文書作成の種類や目的によっては、とても重宝なツールと言えます。……
\onecolumn
以下続く……\\
TeXはスタンフォード大学教授(数学)D.E.Knuth(19388~)による文書整形システムです。ar TeXは大抵「テフ」と読まれています。TeXはワープロのたぐいと言えますが、より正しくは、1つのプログラム言語に近いものです。利用者によるマクロ命令によって機能を拡張することができます。………
\end{document}