% toukei7R.tex
\documentclass[a4paper]{jarticle}
\usepackage{amsmath,amssymb,latexsym}
\pagestyle{empty}
%印刷部分の大きさの指定
\setlength{\topmargin}{-50pt}
\setlength{\oddsidemargin}{-3mm}
\setlength{\evensidemargin}{-3mm}
\setlength{\textheight}{25cm}
\setlength{\textwidth}{14.5cm}
\setlength{\footskip}{6mm}
\begin{document}
\begin{center}
   {\large\bf 推定のレポート問題}
\end{center}
正規分布,t-分布,カイ2乗分布それぞれのパーセント点の値
$z_{\alpha}$, $t_{\alpha}(n)$, $\chi_{\alpha}^2(n)$ については
統計学のテキスト(たとえば「統計学演習」村上正康他著,倍風館,付表203,
205,206ページを参照,あるいはホームページに
http://www.math.s.chiba-u.ac.jp/~yasuda/index-j.htm)
には載っている。パスワードは授業で発表している。


\vspace{2ex}

{\bf 問1.} \hspace{0.5em}
平均$\mu$,分散$\sigma^2$をもつ母集団から,大きさ3の標本$X_1,X_2,X_3$
を得たとき,
$$
\begin{array}{ll}
  T_1 & = X_1, \\
  T_2 & = \dfrac{X_1+X_3}{2}, \\
  T_3 & = \dfrac{X_1+X_2+X_3}{3} 
\end{array}
$$
つぎの問いに答えよ。
\begin{itemize}
  \item[(1)]
   いづれも平均 $\mu$ の不偏推定量であることを示せ。
  \item[(2)]
   これらの分散を計算して,このなかでは $T_3$ が
   もっとも分散が小さくなることを示せ。
\end{itemize}


\vfil

{\bf 問2.} \hspace{0.5em}
\begin{itemize}
 \item[(1)] つぎの命題を示せ。
   $a,b,c > 0, a + b + c = 1 $ のとき,$a^2 + b^2 + c^2 $ が
   最小となるのは, $a= b= c= 1/3$ であることを示せ。
 \item[(2)]
   平均 $\mu$, 分散 $\sigma^2$ をもつ母集団から,大きさ3の標本
   $X_1,X_2,X_3$ を得たとき,
   $ T = a X_1 + b X_2 + c X_3$ 
   が $\mu$ の不偏推定量と
   なるためにはどういう条件があればよいか? 
   またそのとき最小となる分散を計算せよ。
\end{itemize}

\vfil

{\bf 問3.} \hspace{0.5em}
あるクラス$40$人の試験得点分布は正規分布にしたがうという。
得点の平均が$58.2$, 標準偏差は$10.3$であった。
母平均の$90\%$信頼区間を求めよ。

\vfil

{\bf 問4.} \hspace{0.5em}
平均がゼロという正規母集団から大きさ10の
標本$X_i,i=1,2,\cdots,10$
を抽出した。データの和と2乗和を計算すると,
$\sum_{i=1}^{10} X_i = 3.192,
\sum_{i=1}^{10} X_i^2 = 37.983,$
であった。このとき,分散の95\%信頼区間を
求めよ。


\vfil

{\bf 問5.} \hspace{0.5em}
ある集団の喫煙率を調べるために,300人に質問を
したところ,72人が喫煙をしていた。このとき,集団の
喫煙率に対する,90\%信頼区間を求めよ。

\vfil

\end{document}
